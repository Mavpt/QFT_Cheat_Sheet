% -----Theorems-----
\theoremstyle{plain}
\newtheorem{theorem}{Theorem}[section]
\newtheorem{lemma}{Lemma}[section]
\newtheorem{corollary}{Corollary}[section]
\newtheorem{proposition}{Proposition}[section]
\newtheorem{conjecture}{Conjecture}[section]

\newtheorem{identity}{Identity}[section]

\theoremstyle{definition}
\newtheorem{definition}{Definition}[section]
\newtheorem{condition}{Condition}[section]
\newtheorem{problem}{Problem}[section]
\newtheorem{example}{Example}[section]

\theoremstyle{remark}
\newtheorem{remark}{Remark}[section]

%\renewcommand*{\qedsymbol}{QED}

% ----- Blackboard Bold -----
\DeclareMathOperator{\1}{\mathbb{1}}

\DeclareMathOperator{\Ab}{\mathbb{A}}
\DeclareMathOperator{\Bb}{\mathbb{B}}
\DeclareMathOperator{\Cb}{\mathbb{C}}
\DeclareMathOperator{\Db}{\mathbb{D}}
\DeclareMathOperator{\Eb}{\mathbb{E}}
\DeclareMathOperator{\Fb}{\mathbb{F}}
\DeclareMathOperator{\Gb}{\mathbb{G}}
\DeclareMathOperator{\Hb}{\mathbb{H}}
\DeclareMathOperator{\Ib}{\mathbb{I}}
\DeclareMathOperator{\Jb}{\mathbb{J}}
\DeclareMathOperator{\Kb}{\mathbb{K}}
\DeclareMathOperator{\Lb}{\mathbb{L}}
\DeclareMathOperator{\Mb}{\mathbb{M}}
\DeclareMathOperator{\Nb}{\mathbb{N}}
\DeclareMathOperator{\Ob}{\mathbb{O}}
\DeclareMathOperator{\Pb}{\mathbb{P}}
\DeclareMathOperator{\Qb}{\mathbb{Q}}
\DeclareMathOperator{\Rb}{\mathbb{R}}
\DeclareMathOperator{\Sb}{\mathbb{S}}
\DeclareMathOperator{\Tb}{\mathbb{T}}
\DeclareMathOperator{\Ub}{\mathbb{U}}
\DeclareMathOperator{\Vb}{\mathbb{V}}
\DeclareMathOperator{\Wb}{\mathbb{W}}
\DeclareMathOperator{\Xb}{\mathbb{X}}
\DeclareMathOperator{\Yb}{\mathbb{Y}}
\DeclareMathOperator{\Zb}{\mathbb{Z}}

% ----- Calligraphy -----
\DeclareMathOperator{\Ac}{\mathcal{A}}
\DeclareMathOperator{\Bc}{\mathcal{B}}
\DeclareMathOperator{\Cc}{\mathcal{C}}
\DeclareMathOperator{\Dc}{\mathcal{D}}
\DeclareMathOperator{\Ec}{\mathcal{E}}
\DeclareMathOperator{\Fc}{\mathcal{F}}
\DeclareMathOperator{\Gc}{\mathcal{G}}
\DeclareMathOperator{\Hc}{\mathcal{H}}
\DeclareMathOperator{\Ic}{\mathcal{I}}
\DeclareMathOperator{\Jc}{\mathcal{J}}
\DeclareMathOperator{\Kc}{\mathcal{K}}
\DeclareMathOperator{\Lc}{\mathcal{L}}
\DeclareMathOperator{\Mc}{\mathcal{M}}
\DeclareMathOperator{\Nc}{\mathcal{N}}
\DeclareMathOperator{\Oc}{\mathcal{O}}
\DeclareMathOperator{\Pc}{\mathcal{P}}
\DeclareMathOperator{\Qc}{\mathcal{Q}}
\DeclareMathOperator{\Rc}{\mathcal{R}}
\DeclareMathOperator{\Sc}{\mathcal{S}}
\DeclareMathOperator{\Tc}{\mathcal{T}}
\DeclareMathOperator{\Uc}{\mathcal{U}}
\DeclareMathOperator{\Vc}{\mathcal{V}}
\DeclareMathOperator{\Wc}{\mathcal{W}}
\DeclareMathOperator{\Xc}{\mathcal{X}}
\DeclareMathOperator{\Yc}{\mathcal{Y}}
\DeclareMathOperator{\Zc}{\mathcal{Z}}

% ----- Physics -----
\newcommand*{\nord}[1]{%
  {\boldsymbol{:}\mathrel{#1}\boldsymbol{:}}%
}

\newcommand*{\Tordp}[1]{%
  {\symbf{T}\qty[#1]}%
}

\DeclareMathOperator{\DiracDelta}{\mbox{\(\delta\)}}
\DeclareDocumentCommand\Ddelta{}{\trigbraces{\DiracDelta}}
%\DeclareDocumentCommand\Ddelta{}{\trigbraces{\DiracDelta}}
